\documentclass[12pt, a4paper, twoside]{book}

\usepackage{amsmath, amsfonts, amssymb} % Basic math package

\usepackage{ctex} % LaTeX for Chinese

\numberwithin{equation}{section} % Number the equation with the section

\usepackage{graphicx, subfigure} % Used for figures and subfigures
% Example
\begin{figure}[!ht]
    \centering
    \includegraphics[options]{filename}
    \subfigure[caption]{\includegraphics[options]{filename}}
    \caption{Figure}
\end{figure}

\usepackage{enumerate} % Used to empower the enumerate
% Example
\begin{enumerate}[(1)]
    \item This is the first item. 
    \item This is the second item. 
\end{enumerate}

\usepackage[colorlinks, linkcolor = blue]{hyperref} % Adjust properties of hyper-links

\usepackage{url} % Used to enable exterior links
% Example
\url{https://www.baidu.com/}

\usepackage{colortbl, booktabs} % Set the table background color and line style
% Example
\begin{table}[!ht]
    \centering
    \begin{tabular}{c|c|c|c}
        \bottomrule
        \rowcolor[gray]{0.9}
        & Column 1 & Column 2 & Column 3 \\
        \hline
        Row 1 & $a_{11}$ & $a_{12}$ & $a_{13}$ \\
        \hline
        Row 2 & $a_{21}$ & $a_{22}$ & $a_{23}$ \\
        \toprule
    \end{tabular}
    \caption{Table}
\end{table}

\usepackage{geometry} % Used to adjust the page style
\setlength{\parindent}{0em} % Adjust the paragraph indent
\geometry{left = 2cm, right = 2cm, top = 3cm, bottom = 3cm} % Adjust the page margin

\usepackage{framed, xcolor} % Used to create a shade box
\definecolor{shadecolor}{RGB}{237, 237, 255} % Adjust the shade color
% Example
\begin{shaded}
    This is an example of a shaded box. 
\end{shaded}

\usepackage{ntheorem, framed, xcolor} % Used to empower theorem environments (with shaded boxes)
\definecolor{shadecolor}{RGB}{237, 237, 255} % Adjust the shade color
\theorembodyfont{\rmfamily} % The font of the theorem body
\newtheorem{theorem}{Theorem}[section]
\newtheorem{_def}[theorem]{Definition}
\newenvironment{definition}{\begin{shaded}\begin{_def}}{\end{_def}\end{shaded}}
\newtheorem{proposition}[theorem]{Proposition}
\newtheorem{corollary}[theorem]{Corollary}
\newtheorem{lemma}[theorem]{Lemma}
\newtheorem{example}[theorem]{Example}
\newtheorem*{proof}{Proof}
\newtheorem*{solution}{Solution}
\newtheorem*{note}{Note}

% Self-defined problem-solution environment
\newcounter{problemname}[section]
\newenvironment{problem}{\refstepcounter{problemname}\vspace*{0.5\baselineskip}\par\noindent\textsc{Problem \arabic{chapter}.\arabic{section}.\arabic{problemname}} }{\par\vspace*{0.5\baselineskip}}
\newenvironment{solution}{\vspace*{0.5\baselineskip}\par\noindent\textsc{Solution }}{\par\vspace*{0.5\baselineskip}}
\newenvironment{note}{\vspace*{0.5\baselineskip}\par\noindent\textsc{Note }}{\par\vspace*{0.5\baselineskip}}

\usepackage{listings, xeCJK, xcolor} % Used to create coding boxes and output boxes
\definecolor{cmtColor}{rgb}{0, 0.5, 0}
\definecolor{strColor}{rgb}{0.639, 0.082, 0.082}
\definecolor{bgColor}{rgb}{0.9, 0.9, 0.9}
\newcounter{C++}[section]
\newcounter{output}[section]
\lstset{
	% Script
	basicstyle = \footnotesize\ttfamily,	% Basic style
	keywordstyle = \bfseries\color{blue},	% Keyword style
	otherkeywords = {},						% Add other keywords
	deletendkeywords = {},					% Keywords to be deleted
	commentstyle = \color{cmtColor},		% Comment style
	stringstyle = \color{strColor},			% String style
	extendedchars = true,					% Enable extended character sets
	breaklines = true,						% Determine if a line would break when too long
	breakatwhitespace = false,				% The line only breaks at white-space
	keepspaces = true,						% Keep spaces in text
	showspaces = false,						% Show spaces everywhere to add particular underscores
	showstringspaces = false,				% Show spaces in string to add particular underscores
	showtabs = false,						% Show tabs everywhere to add particular unders
	tabsize = 4,							% Size of tab
	flexiblecolumns = true,					% Enable flexible columns
	% Line number
	numbers = left,							% The position of the line number
	numberstyle = \scriptsize\color{black},	% Style of the line number
	numbersep = 8pt,						% Distance between the number and the coding block
	% Caption
	captionpos = b,							% The position of the caption
	% Escape
	escapeinside = {<@}{@>}					% Symbols inside will be compiled as LaTeX code
}
\lstnewenvironment{Cpp}[2]{
	\stepcounter{C++}
	\renewcommand*{\lstlistingname}{Code}
	\renewcommand{\thelstlisting}{C.\arabic{section}.\arabic{C++}}
	\lstset{
		% Language
		language = C++,						% Language of the code
		% Appearance
		frame = shadowbox,					% Frame style of the coding block
		backgroundcolor = \color{bgColor},	% Background color of the coding block
		framerule = 0.3pt,					% Width of the border
		rulecolor = \color{black},			% Color of the border
		rulesepcolor = \color{gray},		% Color of the shadow
		% Caption
		caption = {#1}						% Caption
	}
	\label{#2}
}{}
\lstnewenvironment{Cpp*}[1]{
	\renewcommand*{\lstlistingname}{Code}
	\renewcommand{\thelstlisting}{}
	\lstset{
		% Language
		language = C++,						% Language of the code
		% Appearance
		frame = shadowbox,					% Frame style of the coding block
		backgroundcolor = \color{bgColor},	% Background color of the coding block
		framerule = 0.3pt,					% Width of the border
		rulecolor = \color{black},			% Color of the border
		rulesepcolor = \color{gray},		% Color of the shadow
		% Caption
		caption = {#1}						% Caption
	}
}{}
\lstnewenvironment{Output}[2]{
	\stepcounter{output}
	\renewcommand*{\lstlistingname}{Output}
	\renewcommand{\thelstlisting}{O.\arabic{section}.\arabic{output}}
	\lstset{
		% Appearance
		frame = ltrb,						% Frame style of the coding block
		backgroundcolor = \color{white},	% Background color of the coding block
		framerule = 0.3pt,					% Width of the border
		% Caption
		caption = {#1}						% Caption
	}
	\label{#2}
}{}
\lstnewenvironment{Output*}[2]{
	\renewcommand*{\lstlistingname}{Output}
	\renewcommand{\thelstlisting}{}
	\lstset{
		% Appearance
		frame = ltrb,						% Frame style of the coding block
		backgroundcolor = \color{white},	% Background color of the coding block
		framerule = 0.3pt,					% Width of the border
		% Caption
		caption = {#1}						% Caption
	}
}{}
\makeatletter
\renewcommand{\lstlistlistingname}{List of codes and outputs}
\renewcommand{\l@lstlisting}[2]{\@dottedtocline{1}{1.5em}{4em}{#1}{#2}}
\makeatother

\usepackage[most]{tcolorbox} % Used to create fancy theorem environments
\usepackage{ntheorem, framed, xcolor} % Used to empower theorem environments (with shaded boxes)
% Define colors
\definecolor{thmColor}{rgb}{0.486, 0.533, 0.439}
\definecolor{lemColor}{rgb}{0.529, 0.580, 0.655}
\definecolor{corColor}{rgb}{0.635, 0.600, 0.541}
\definecolor{defColor}{rgb}{0.580, 0.325, 0.341}
\definecolor{exaColor}{rgb}{0.337, 0.518, 0.569}
\definecolor{shadecolor}{RGB}{237, 237, 255}
% Define textbox style
\newtcolorbox{textbox}[2]{
	enhanced,
	breakable,
	colback = white,
	colframe = {#2},
	title = {#1},
	fonttitle = \bfseries,
	attach boxed title to top left,
	boxed title style = {
		size = small,
		colback = {#2},
		colframe = {#2},
	}
}
% The font of the theorem body
\theorembodyfont{\rmfamily}
% Introduce a new counter
\newtheorem{thm}{Theorem}[section]
% Define the theorem environment
\newenvironment{theorem}[1][]{\refstepcounter{thm}\begin{textbox}{定理 \thethm\ifthenelse{\equal{#1}{}}{}{(#1)}}{thmColor}}{\end{textbox}}
\newenvironment{definition}[1][]{\refstepcounter{thm}\begin{textbox}{定义 \thethm\ifthenelse{\equal{#1}{}}{}{(#1)}}{defColor}}{\end{textbox}}
\newenvironment{corollary}[1][]{\refstepcounter{thm}\begin{textbox}{推论 \thethm\ifthenelse{\equal{#1}{}}{}{(#1)}}{corColor}}{\end{textbox}}
\newenvironment{lemma}[1][]{\refstepcounter{thm}\begin{textbox}{引理 \thethm\ifthenelse{\equal{#1}{}}{}{(#1)}}{lemColor}}{\end{textbox}}
\newenvironment{example}[1][]{\refstepcounter{thm}\begin{textbox}{例 \thethm\ifthenelse{\equal{#1}{}}{}{(#1)}}{exaColor}}{\end{textbox}}
\newtheorem*{_proof}{证明}
\newtheorem*{_solution}{解}
\newtheorem*{note}{注}
\newenvironment{proof}{\begin{_proof}}{\hfill{$\Box$}\end{_proof}}
\newenvironment{solution}{\begin{_solution}}{\hfill{$\Box$}\end{_solution}}
% Automatically generate an index of the theorems
\usepackage{etoolbox} % Used to define array of strings, maybe not necessary
\def\envname{}
\newcounter{rownum}
\newcommand{\newtheoremindex}[1]{\def\cntname{lbl#1}\newcounter{\cntname}}
\newcommand{\addtext}[1]{
    \def\cntname{lbl\envname}
    \def\cmdname{txt\envname}
    \stepcounter{\cntname}
    \global\csdef{\cmdname\arabic{\cntname}}{#1}
}
\newcommand{\gettext}[1]{\csuse{txt\envname#1}}
\newcommand{\addlabel}[1]{
    \def\cntname{lbl\envname}
    \global\csdef{\cntname\arabic{\cntname}}{#1}
    \label{#1}
}
\newcommand{\getlabel}[1]{\csuse{lbl\envname#1}}
\newcommand{\maketheoremindex}[2]{
    \setcounter{rownum}{0}
    \def\cntname{lbl#1}
    \global\def\envname{#1}
	\begin{tabbing}
		\textbf{#2编号}\hspace{1cm}\=\textbf{页码}\hspace{1cm}\=\textbf{说明}\\
		\whileboolexpr
			{ test {\ifnumcomp{\value{rownum}}{<}{\arabic{\cntname}}} }
			{\stepcounter{rownum}\ref{\getlabel{\therownum}}\>\pageref{\getlabel{\therownum}}\>\gettext{\therownum}\\}
	\end{tabbing}
}
\newtheoremindex{thm}
\newenvironment{theorem}[1][]{\global\def\envname{thm}\addtext{#1}\refstepcounter{thm}\begin{textbox}{定理 \thethm\ifthenelse{\equal{#1}{}}{}{(#1)}}{thmColor}}{\end{textbox}}

% Using EnglischeSchT script font style
\newfontfamily{\calti}{EnglischeSchT}
\newcommand{\mathcalti}[1]{\mbox{\calti{#1}}}
\newcommand{\mathcaltibf}[1]{\mbox{\bf\calti{#1}}}